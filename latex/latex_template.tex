\documentclass[a4paper, 12pt]{article}
\usepackage{graphicx}

\usepackage[utf8]{inputenc}
\usepackage[magyar]{babel}

% a szép matematikai szimbólumokért
\usepackage{amssymb}
\usepackage{amsmath}

% eps formátumú ábrák --> pdflatex fordításhoz!!
\usepackage{epsfig}

%%%%%%%%%%%%%%%%%%%%%%%%%%%%%%%%%%%%%%%%%%%%%%%%%%%%%%%%%%%%%%%%%%%%%%%%%%%%

\author{Bíró Gábor}

\title{A kutatómunka információs eszközei}

\begin{document}

\begin{titlepage}   

\maketitle

\date

\end{titlepage}

\newpage

%%%%%%%%%%%%%%%%%%%%%%%%%%%%%%%%%%%%%%%%%%%%%%%%%%%%%%%%%%%%%%%%%%%%%%%%

\section{Bevezetés}
\label{sec:bev}
Itt egy hangzatos bevezetés szerepel.

\subsection{Egy alfejezet}
\label{subsec:alfejezet}

Ez meg egy alfejezet.

\subsubsection{Alalfejezet}
\label{subsubsec:alalfejezet}

Ez pedig egy alalfejezet. Nevezetesen a \ref{sec:bev}. bevezetés után kezdődő \ref{subsec:alfejezet}. alfejezetnek az \ref{subsubsec:alalfejezet}. alalfejezete.

\begin{equation}
a+b=c
  \label{eq:eq1}
\end{equation}

Ez volt az egyenlet. 

\subsubsection{A \LaTeX\ dokumentum lefordítása}
A fordításhoz végrehajtandó lépések:

\begin{itemize}
  \item[a)] van \textit{eps}-től eltérő képformátum (\textit{png, jpg} vagy más) a dokumentumban:
  \begin{enumerate}
    \item pdflatex fajlnev.tex
    \item bibtex fajlnev.aux
    \item pdflatex fajlnev.tex
    \item pdflatex fajlnev.tex
  \end{enumerate}
  \item[b)] nincs \textit{eps}-től eltérő képformátum a dokumentumban:
  \begin{enumerate}
    \item latex fajlnev.tex
    \item bibtex fajlnev.tex
    \item latex fajlnev.tex
    \item latex fajlnev.tex
    \item dvipdf fajlnev.tex
  \end{enumerate}
  \item harmadik opció nincs. Ezzel csak demonstrálom, hogy mi az \texttt{itemize} alapértelmezett szimbóluma.
\end{itemize}

\end{document}

Ez pedig már nem fog megjelenni, mivel az end document után vagyunk.
